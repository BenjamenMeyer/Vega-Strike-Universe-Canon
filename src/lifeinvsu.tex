\label{chapt:livinginVSU}
\section{How to think about various groups}
\label{sec:groupintuitions}
This subsection isn't designed to give you all of the info on a given
group (check out one of the appendices for that). What this subsection
{\bf is} designed to do is to give an idea of what frame of mind one
should be in when considering a particular group in the VS universe
and how that group relates to others, themselves, and their
surroundings. Frequently, I will list other groups or entities that
come to mind when I am pondering a given VSU group. However, please do
not confuse influences with instantiations. If, for example, the
Moties from {\em A Mote in God's
Eye}~\cite{NivenPournelle-MoteInGodsEye} are listed as an influence in
the development of the Aera, it does {\bf not} mean that the Aera are
necessarily similar to the Moties. Rather, it means only that the
Moties were on my mind when developing the Aera, and are perhaps a
useful point of reference to be familiar with for framing one's own
pondering of the Aera. Some listed influences will, in fact, be
strongly negative in their correlation to the actual nature of the
group (e.g. the Borg~\cite{StarTrek-Borg} may be listed as an
influence because I am actively trying to ensure that a group {\bf
doesn't} get considered as similar.

\subsection{Humanity}

One might think this part the least necessary group to consider, but
it's actually the most tricky. As humans, we've got a pretty good idea
of how humans operate. As we drift away from our current conceptions,
either confusion or disbelief can ensue. And we're going to have to
drift away a bit here, both for groups with trans-humanist directions
or aspirations, and simply because of the 1200 year gap between
ourselves and most of the VS cultures.

\begin{itemize}
\item {\bf Rule number one} - These are not the people around you. At least, many
of them aren't. The people of the 33rd century, by and large, bear
less resemblance to you than you do to a 10th century peasant - this
much is to be expected.

\item {\bf Rule number two} - Some of these "people" REALLY AREN'T the
people around you - at all. It's not just the cultural gap. Thinking
about the Purists as fairly normal, if scared, people, the Unadorned
as somewhat nutty religious people, the Forsaken (even more like
modern man than the Purists) as bitter people, the Highborn as
self-absorbed (and perhaps mildly self-deluding) people and the
Merchants as greedy people can lead to somewhat reasonable grips on
how these groups operate - they are, at heart, fundamentally still
human, if culturally distinct from today's climate. Even the
Mechanists can be superficially grokked by starting with a zealous
level of self-hatred directed at the limitations of their human
bodies. However, thinking about the Andolians or Shapers as just human
will delude you, and lead your conclusions astray. They are not yet
alien, but they are intensely foreign to the humanity we are familiar
with; they no longer think like us. The Andolians, collectively,
haven't forgotten anything meaningful for over 900 years. Each
generation grows up with immediate and nearly innate access to more
information than each generation before. They are connected, not just
in the simple physical sense of their link, but also in social senses
that modern man simply isn't. They don't think about self and the
other in the same way we do - they can't. The Shapers have adult minds
by the age of 7 and even their dullest healthy member surpasses most
modern humans. They are a society whose rate of idiocy, mental
defects, physical defects, malnutrition and insufficient pre-natal care
is so microscopic, their disease rate so low, that one it it suffices
to think of them as a post disease, post illness, post weakness
society. Theirs is a society of extreme individualists that runs
smoothly because they're all up on the game that's being played -
duping the Shaper electorate makes bribing the Supreme court look like
something a drooling infant could accomplish by accident. We share
more genes with the SuSims than with the Shapers. They are not gods or
demigods, or any such thing, but to think of them merely as human, is
to do them insufficient justice.

\item {\bf Rule number three} - The "Purist/Luddite" test: while you need
not agree with the eponymous groups, if you can't understand on a
permeating, gut level why these groups are so obsessed with bounding
what constitutes humanity and what it means to lead a human life, then
you don't yet understand the "humans" of the 33rd century that inhabit
the VS universe.
\end{itemize}
Key differentiable human groups include:
\begin{itemize}
\item Andolians

It would perhaps be inaccurate to say that the Andolians are actually
friendlier than the other major meme-groups. More accurate would be to
say that they are more tolerant, as much because they can afford to be
as because it aligns with their outlook. They are, however, often seen
as patronizing or even condescending in their tolerance of other
groups. This is still seen as preferable to the outright disgust,
hatred, or dismissal that can often be experienced between the
meme-groups. The Andolians often refer to each other with sibling
terminology, the Klk'k, even the non-linked, with diminutive sibling
references (bro-chan, sis-chan, etc.), non-Andolian humans in the
protectorate as "steps" or "steppers", the Purth as "little ones" (an
ironic touch, given that the Purth are extremely large), and
non-protectorate humans as "cousins". Such references, however, are
made only in casual discourse, and in generally unambiguous fashion,
with actual relations being made pointedly clear.

\item Forsaken
\item Highborn
\item LIHW
\item Luddites
\item Mechanists
\item Merchants
\item Purists

The Purists and Andolians are the most {\emph POPULOUS} human
populations, but the Purists are limited as an economic power and
industrial power. Purist growth is unmanaged, their infrastructure is
uneven at best, and their governments are varying degrees of corrupt,
incompetent, or overwhelmed. Constantly looking backward, they stumble
often as they advance toward into the future.

The Purists and the LIHW both live in some degree of economic
servitude to the Merchant groups, and the outsourcing of their needs
for capital vessels is business as usual in their governmental
dealings. It's a far easier sell to their constituents than attempting
the investments needed to build sufficient infrastructure to produce
competitive capital vessels of their own design.

To contrast tersely:\\
The Shapers have the most productive individuals.\\
The Andolians have the most productive populations.\\
The Mechanists have the most per capita space-infrastructure.\\
The Unadorned have the largest per capita research investments.\\
The Merchants have the most trade volume.\\
The Highborn have the most political leverage over the Merchants.\\
The Purists have the most people.\\
The LIHW have the most diverse ideological viewpoints.\\

Recall that the Purists ended up with Earth\\
A) after everyone else had left\\
B) after a period of nearly abject governmental and economic collapse

Likewise, recall that the Purist philosophy is less cohesive than that
of some of the other meme-groups. The rise of the Purist movement to
dominance on post nano-plague earth was largely fueled by a scared and
angry counter-reaction to trans-humanism rather than by a well-defined
central message. Though they have since matured ideologically, their
roots as an initially negatively defined world-view still shows
through.

\item Shapers
\item Unadorned
\end{itemize}

\subsection{The Klk'k}

They're wisenheimers, to a degree. Their sense of humor permeates
their civilization more so than ours, making for odd juxtapositions,
such as it being entirely appropriate to be cracking jokes while
fighting, murdering, or engaging in serious policy decisions. The key
to thinking about the Klk'k is this - as much as they may seem to have
progressed along remarkably parallel lines, they're still aliens. As
SF author Gregory Benford once said, "the thing about aliens is,
they're alien." The Klk'k are enough like us, compared to all of the
other aliens, and they can work with us, that we keep wanting them to
be like us and expect them to be like us - but they aren't like us,
and it's always disconcerting when they prove it. One must imagine
asking a Klk'k why they have just done something, having them explain
in what appears to be a rational fashion, and still just being
dumbfounded as to why they did what they did - between differences in
axiomatic values and divergence in the nuances of the explanation it
just wasn't the same way of thinking about the situation, and thus
they arrived at a different outcome.

\subsection{The Aera}

The Aera got the short end of the stick - they drew the bad lot in the
running for "butt of cosmic joke" (perhaps they failed to
appropriately bribe the AUTHORS). Their planet was unpleasant, their
position in the jump network was supremely non-optimal, their timing
was poor and made even worse by the fact that they didn't know that
everyone else was going to run out of real estate soon enough
anyway. They aren't boogie-men, they aren't monsters, they aren't
ravenous alien invaders. They are an abused and shortchanged group
looking to survive in a universe that has repeatedly shown itself
uncaring to their existence. If the Klk'k disturb us when we are
reminded that they are unlike us, the Aera disturb us most when we are
forced to realize that we are not as different as we might like to
think, beneath bodies that each considers extremely ugly. Their
viewpoint tends to be colored by suspicions and certainties of
antagonism, but these are the result of a profoundly guarded outlook,
rather than the delusions of a human paranoid. Aerans are actually
quite distinct as individuals, but their fundamental pack and
abstracted pack loyalty structures allow them to operate cohesively in
groups in a manner that seems far more lockstep, frighteningly
authoritarian, and homogeneous to a human observer than it actually
is. The individual is celebrated post-facto. A life's accomplishments
cannot adequately be judged until that life is completed, from an
Aeran perspective. Don't think of the Aera as bad, as evil, or as
inherently inimical to the other races - this would be a miscarriage
of justice, and not even an oversimplification, but an
untruth. Rather, empathize with their miserable initial situation,
even if the only sane way for humanity to deal with them, alien and
resolute as they are, is to shoot back at them.

\subsection{The Rlaan}

The Rlaan are intensely alien. If the Klk'k are frustratingly alien,
and the Aera are at times painfully alien, then the Rlaan are
mind-bogglingly alien. They are, in fact, so alien that we can't
really understand how alien they are, because we can't identify what
in their behaviors is just complex and what is derived from more
fundamental differences. The scale just saturates at some
point. Neither they nor we really understand one another, and we
merely have gotten good at pretending. Take their civilian/worker -
defender split; they view any individual capable of willingly killing
a worker the way we'd view someone who liked to feast upon a raw,
unborn fetus, freshly cut out from its mother's womb, while wearing
its freshly vivisected infant siblings as shoes so that his feet won't get
cold while he's carving a scarf out of the mother's back and humming
along listening to the screams of the father as he slowly slides down
an impaling post. We have nothing remotely comparable to that -
nothing. They experience the world in parallel layers at a time, in
sight, in sound, in thought, decomposing their reality into fragments
and piecing it back together. They live for hundreds of years, but even
if that's actually a fairly short time for life at their temperatures,
they don't have any sense of individual urgency in their life. While
the Aera are vibrant individuals underneath the firm veneer of their
society, the Rlaan are, by and large, extremely similar creatures
underneath the cloak of chaotic motion that constitutes fair portions
of their society. Rlaan populations are large enough that, even with a
much smaller standard deviation, there are exceptional individuals,
but most Rlaan, especially the workers, are remarkably interchangeable
despite their differences - this is not because they do not
differentiate themselves significantly, but rather because they
differentiate themselves in ways that are reversible. Underneath
whatever they are currently doing and believing, Rlaan minds seem to
function in remarkably similar fashion to one another. A conversion to
a new mindset can make the average Rlaan a good stand-in for any
another.

Humans, however, do not often interact with the uninteresting Rlaan,
and it greatly colors our perceptions of them. Only those Rlaan
trusted with having inklings of how other minds functions are allowed
to be their diplomats. The anthrophilic Rlaan-Briin are vital to
increasing cultural understanding, but they're a distinct minority
among the Rlaan, and those, even of the Rlaan-Briin, who are capable
of moving toward "foreign" from "alien" are an even smaller
minority. We, on the other hand, have never moved from "alien" toward
"foreign" for them on our own. It is only as the result great
assistance and analysis from AIs and PAIs that we can now convince
ourselves that the Rlaan receive messages truly similar to what we
believe we are sending them.

\subsection{The Uln}

Boorish, feudal, and seemingly anachronisms, the Uln are alien, but
surprisingly uncomplicated to the degree that our interactions are
unsubtle. They are willing and well practiced in mimicking aspects of
the civilizations and societies of those they deal with, and, though
it masks deeper misunderstandings and differences, this allows them to
at least appear less alien than they truly are in the context of
particular dealings with them. They are, in many ways, a deeply
insecure people, given to grandiose displays of overcompensation.

\subsection{The Shmrn}

The Shmrn are fundamentally depressed and fatalistic. Their lives are
short, and their existence tends toward one of chronic mild
discomfort. We created them, and have left so deep an imprint upon
their psyche that we are not entirely wrong to anthropomorphize some
of our assumptions with respect to their internal mental states.

\subsection{The Dgn}

Though from the same stock as their Shmrn brethren, the Dgn have been
far more effectively subjugated by their Shaper masters. They do not
welcome their condition, but do not find it particularly irksome.

\subsection{The Saahasayaay}

The Saahasayaay thirst for violence and consumption is best described
in terms of lust. Their embrace of violent means to achieve ends may
lead one to believe them to be hedonistic sadists, but that would be
somewhat askew. They do not perceive their domain to be that of pain
or suffering, but of death. All else is incidental, except that it
reflects their belief in ultimate dominion over life. With their own
peculiar degree of immortality, they are consumed by their fascination
with the termination of existence. The Rlaan have often regretted not
leaving them to rot on their stagnant stone-aged planet.

\subsection{The Purth}

\subsection{The Alphans/Betans}

\subsection{The Ancients}

\subsection{The TWHON}
The most important thing to remember about the TWHON is what they are
not: gods. The second most important thing to remember about the TWHON
is that there was really only one of them.


\section{Economics and Day-to-day Living}

Lots of jobs in construction. Constant expansion.

\subsection{Transportation and Craft Ownership}

In the VSU, if you're in space, the odds are you're a passenger on a
dedicated passenger craft, be it public or private in nature. Pilots
are relatively few in number. In stark contrast to, for instance, the
Star Wars universe, personal spacecraft ownership in the VSU is
low. Not only are costs high, but security issues (even small craft
can be highly destructive) limit the number of ``tramp freighters'' of
dubious origin wandering about the VSU. Craft have official and
well-distributed IDs backed by some sizable political or economic
entity, or else they tend to get shot at. While these IDs and ID
checks aren't immune to forgery or to corruption allowing less than
reputable characters to get their hands on valid IDs, they do limit
anonymity. As for costs, the cheapest spaceship is still going to cost
more than many homes on well developed worlds.

% LocalWords:  Mechanists Andolians SuSims Klk'k Andolian Purth LIHW Benford
% LocalWords:  wisenheimers Aera Aerans Aeran Rlaan PAIs Uln Shmrn Dgn Alphans
% LocalWords:  Saahasayaay Betans TWHON
