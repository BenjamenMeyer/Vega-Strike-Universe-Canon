\label{chapt:PCplots}
\section{Dramatis Personae}
\begin{itemize}
\item Deucalion (Player Character)

Deucalion is a human, albeit of heavily modified genetic stock and
augmented with numerous implants (the latter being the norm among the
citizens of the Protectorate, and the former being a somewhat
distinguishing, albeit hardly unique, feature). He was raised from
near-infancy by Klk'k on Ktah after the untimely demise of the craft
carrying him and, so far as could be discovered, anyone with any claim
to him.  He remained on Ktah throughout his youth, though visits
alongside his adoptive family to other Protectorate worlds were not
unheard of, if infrequent.  His first long term separation from Ktah
and family came when he attended First University on Kubernan. Having
completed his studies in Computer Science, Historical Analysis, and
Xenolinguistics, he returned to Ktah for his Universal Service
requirement, his aptitude exam placing him into Officer's training at
APSWAK (Andolian Protectorate Space Warfare Academy at Ktah), where he
was trained as a pilot.

At APSWAK, Deucalion met Lauktk, a Klk'k flight mechanic who remains
assigned to Deucalion's wing when they both go into active duty.  They
become very close friends, and Lauktk later becomes a bond-mate of
Deucalion's adoptive sister.  Upon discharge, Deucalion is recruited
to return to APSWAK as a flight instructor, while Lauktk takes on a
position as a starship mechanic working for the Protectorate Fleet
shipyard orbiting Ktah.  After a few years of saving up, Lauktk has
accumulated enough resources to purchase his own vessel, and contacts
Deucalion, who agrees to pilot the vessel.

At the start of the game Deucalion is ~28 years of age.  He has left
his position as a flight instructor at APSWAK to pilot his best
friend's ship, with the first launch of the vessel occurring only a
deci-year ago. However, only a week into their travels, a Luddite
attack crippled the ship and forced an emergency landing on an
underdeveloped colony world in Cephid 17 that neither Lauktk nor much
of the cargo survived. A few weeks later, ownership and insurance
issues resolved, the ship has been moved and repaired, and what
systems that time and money could afford have been replaced. Likewise,
in large part due to his genetic modifications, Deucalion has mostly
healed physically, if not mentally.

Finally ready to return to space, he carries with him the cremated
remains of Lauktk and the far heavier burdens of guilt and the
untimely broken dreams of the being that was his closest friend.

\item Lauktk (deceased) (Klk'k) 

Deucalion's best friend, flight mechanic and
companion throughout their stint in the Protectorate military.

\item Mai (Klk'k)

Deucalion's adoptive sister, a Klk'k.

\item Mirabel (Human, designer genome)

One of Deucalion's former squad mates. 

Hers is a striking figure, a dark-elf visage conjuring the essence of
a Boris Vallejo piece writ in flesh. From indigo skin to hair
equivocating with metallic sheen between purple and white, she is
clearly a work of gene-smith's art. There is, however, neither love
nor thanks to be had for her crafters. She was commissioned by sexual
sadists with pockets deep enough for expensive toys and fashioned by a
business group willing to sell scruples and humans alike. Fortunately
for her, she was liberated from the facility where she was being
raised before her erstwhile owners could collect her. She, along with
a number of other constructed beings gained their freedom when the
complex was raided during the Blooding of the Purth some thirty years
ago. Mirabel has spent her entire adult life in service to the
Andolian Protectorate. She first met Deucalion during his UniServe,
when they were both posted to the same vessel.
\end{itemize}

{\bf Minor Characters}

\begin{itemize}
\item Jenek (Human) 

\item Simon XII
\end{itemize}

\section{Scenes out of time and sequence}

\begin{itemize}
\item Deucalion and Mirabel after Lauktk's death

"Mommy Dearest": AP slang/jargon for the Andolian Protectorate Ministry of Defense.

Deucalion is sitting alone at a table in a spaceport bar/restaurant,
silently watching the flow of patrons, passively absorbing nearby
conversations, and working with only mild interest through his
meal. Mirabel walks in and sits down at his table.

Deucalion (without particularly looking up): "I see you were on
Kubernan when the Aera decided to wander coreward - or was it
Hephaestus? Either way, not like you to be so far from the front."
(Turning his head up to look directly at her) "Less like you to be
here now."

Mirabel: "Is it that obvious?"

Deucalion (with the first signs of friendliness to his response):
"No." (Smiling) "Only someone already familiar with how quickly your
hair reaches shedding length could tell how long it's been since your
sacrifice to the guardians of sterility."

M: (sincerely) "How - how are you... ?"

D (shaking his head): "I've been better, but I'm sure Mommy Dearest
didn't send you here for a hand-holding session."

M: (More formally, with slight avoidance) "It was on the way. And I do need an answer - how are you?"

D: "As a friend, or as an officer?"

M: "My time is too short right now to be anything but both."

D: "Then I'm not ready to say yet. I need to work through this before
I can tell you which of the other sides I'm going to come out on. The
timing is lousy, but, some things set straighter and more quickly than
others."

M: "I understand."

D: "Do you?"

M: "We miss him too."

D: (with a lopsided grin) "Sometimes I think it's only when you say 'we' that I can really trust you mean 'I'."

M: (With wry humor) "{\em WE} understand. But the timing is all the more unfortunate..." (getting up from the table) "When you return to Ktah, I'll be waiting for you. We'll talk then."

D: "No, you won't. You'll be there, but you won't be waiting, at least, not waiting for me."

M: "You know I - you know, Deucalion, you KNOW." (brief pause) "I have to go. "

D (as she turns to leave): "Mirabel - it was good to see you."

M: (turning back) "And you. But, be well - I'd prefer to remember you when you smiled more easily." (Mirabel smiles slightly, and leaves)

\item Deucalion and K'hoeama, when Deucalion first approaches his ship in the hangar

K'hoeama: Hey, you, HU-NAM! Nice tats. (Points at the Llama) That our ride?

Deucalion: Doctor K'hoeama, I presume.

K: Please, call me Larry. No one else does, but I'm a Klk'k of leisure and I've been told it suits me.

D: ...

D: You do realize I'm not even going to try to figure out what century's archives I'd have to search to comprehend that joke

K: Bah! Lauktk always said your sense of humor was underdeveloped. Still -- (K'hoeama flashes a toothy smile and embraces Deucalion, arm to arm) 

K: It's good to see you up and hopping. From the crash report I expected something worse, but you look at least half as hale as when I saw you last at the bond-set festivities. 

\item Interlude 0: Dreams and Nightmares of Our Fathers Passed

{\center {\it Turn and face the strange

Ch-ch-changes

Just gonna have to be a different man.}}

\centerline{David Bowie - Changes}

[Deucalion] [overlap with credits?? - an interlude?]

Memories. Old. Older. Ancestral. External.

They taste different.

Dreams, like memories, passed down from parent to child. External, yet encompassing - when co-opted, personal.

Dreams. False memories, near memories, fantasies pleasant, surreal - tasting of incompleteness.

The Klk'k do not dream, their unconscious landscapes barren, the troubles of their worlds processed only in lucid detail.

I must have confused them - scared them perhaps.

Nightmares. 

External, induced

Remembered, relived

Archetypal, developmental

Abstract.

Remembered, relived - recurring

A personal taste

My nightmares have always tasted of memory.

My memories

EVENTS IN INTERLUDE 0

Deucalion recovers from the events that killed Lauktk

Aera invade Forsaken space
\end{itemize}
\subsection{Intro Monologue}

[A Dead Man's Ship (end of interlude 0)] [A quiet shape is sitting
towards the end of a not particularly crowded bar.  There is a quiet
hum of lazy activity, and the bar is swaddled in the awkward grays of
artificial twilight. Despite an odd hue to its skin, perhaps the
legacy of a Shaper ancestor or the enduring design of some
gene-smith's art, it is clearly human, and clearly lost in
thought. Two glasses sit in front of him, one an off-world ale, eerily
beautiful in the subtlety of its contrasts with those of Terra, the
other a glass of ice water thinly disguised with a mint leaf. Both
glasses are untouched, both literally and figuratively, while the man
stares somehow both at and through them all at once, as if trying to
will them to complete the act of drinking on their own, or demanding
that they return his gaze with submission. The vaguely confused
motions of the barkeep, unsure now of how to serve him, break his
brief trance. The man's shape, the complete discipline of his
stillness only apparent with the liquid nature of his motion, twists
to direct his dark eyes up and across the marble slab, fixing restless
on the earnest host. A slight curl appears at the edge of the man's
lips. The glimmer of a compassionate smile, a subtle smirk born of
hubris, or the joy of a man rescued from his own mind, the expression
is enigmatic, unrevealing, and ephemeral. The mouth opens again and he
begins to speak, directing his monologue towards the barkeep, although
it is not clear that the audience he craves includes anyone beyond
himself.]

"It was a deci-year ago today - well, day/night cycles being skewed -"

A momentary pause as he breathes while consulting his data-link.

"No, still today."

He smiles genuinely at the barkeep now, but it quickly melts as his
face returns to a more melancholy expression. In a tone laced with
moments of nostalgia and of emptiness, he continues.

"It was a deci-year ago today. We were in a bar, a lot like this one,
in the spaceport district outside the capital on Ktah - maybe a little
busier." He looks down at the two glasses. "Same drinks. Always the
same drinks. He - Lauktk - he wouldn't ever order anything else. Said
that he and his ancestors had been drinking this brew and its
ancestors since back when the monkey-boys had been kind enough to
relegate themselves to one planet. Said that if the flavor has never
been quite the same since the Lightbearers nuked some of the best
crop- land into oblivion, then at least it makes clear that, in
comparison to meeting humanity, drinking Oolak'kl is not too bad for
one's health. I... well, with my metabolism - there's never been much
point in alcohol consumption, at least not in such small quantities."
He glances briefly at the contrasting colors of his hand and the
bar. "I can't speak for what the flavor may have once been, but I must
admit I find the current one equally unappealing everywhere I've tried
it.  Admittedly, and this is no offense to your fine establishment,
shipping costs being what they are, I can't say the price is the same
off Ktah as on." He looks down towards the one empty barstool to his
left and slows slightly. "But that doesn't really matter now does it?
It doesn't really matter at all.

"He'd taken me out to celebrate. In a few hours, he'd be living his
dream. The deed transfer had finalized, launch inspections had passed
- he had a ship.  After years of mucking around with ships that only
came to him in sickness and left his hands the moment he had restored
them to vibrant health, he had a ship all his own." His demeanor
intensifies. "You can't know what it meant to him - his family had
been sailors, captains, explorers, and merchants since the Klk'k age
of sail. His own ship - it wasn't just a dream, it was a birthright
delayed only by economics and circumstance. It wasn't about money - he
was a starship mechanic working for the Protectorate at the Ktah
shipyards, I'd gone back to the academy and was working as a flight
instructor - it wasn't about acquiring some status symbol - he put
every credit he'd saved into that ship. It was about freedom, his
freedom to sail a new sort of sea, and I was going to help him. I was
going to helm that ship wherever his freedom took him. Even I can't
claim to know what it really meant to him, and I knew him as well as
any man could. He was my brother in arms. He was a bond-mate to my
sister - I remember the first time I introduced..." His words trail
off beneath a frigid gust of mental anguish. "I haven't been able to
see her in person yet - yes of course I've messaged, but you see - you
have to understand I couldn't leave, I couldn't...  the dream is still
here... you have to understand, if I could have..."

The loss of composure ends even more abruptly than it began, emotions
submitting again to a mind well practiced in the arts of control. "We
met during our UniServe. He'd come to see the human who was making a
run at top rank in Amakakt at APSWAK. Those were the salad days of
blissful denial, when, somehow, the Rlaan-Aera conflict that was
boiling over next to us kept us calm and cool behind the razor-drawn
wall of political detachment from the years of slaughter that were to
come. Diversions like Amakakt were even more important then; we all
had a fair idea of what it would mean when war galloped across our
borders, but we knew the odds were against it happening before our
tours were up. It wasn't a denial born of callousness to our younger
brethren who would find themselves in the positions we'd vacated in
relative safety - we just couldn't spend a decade or so brooding,
waiting for the nigh-inevitable. The Andolian spirit of counter-
empirical-tainted optimism has always saturated the Protectorate, even
if I could never embrace it the way he did, but much as I'd like to, I
can't blame that or them, or anything so simple for his fate, even if
blame had ever been something I desired. No, for Lauktk, a society
smothered in optimism was a boon; it fueled the infectious energy
intrinsic to his demeanor, intrinsic to his existence. He was
irrepressible and genuinely funny - even the Purth thought so - I
liked him right from the start. We were friends right up until we
completed our stint at the academy and shipped out - good friends,
even if our different specializations meant we didn't see each other
all that often.

"When we found we'd be serving together on the same ship, we both were
surprised. When we found out that he'd actually be my flight mechanic,
I was...  I felt comforted somehow, to know that the person upon whom
my safety in some large part depended on was someone I knew beyond the
casual connections provided by our links. It was a fairly small crew,
and most of us were there for several years, so we all got to know
each other more than most Andolian flat mates, but the social
structure is never flat - there are always some people you click with
more than others.  The details are both too many to recount and too
meaningless without context, but by the end of that tour, we were
brothers in all but birth.  We'd been through... we'd been through a
lot together - you'll just have to trust me that it's less cliche than
classified." He pauses, asking some question of himself that remains
unspoken, and, the question seemingly answered, he continues.

"Mai, my sister - that was always a disappointment for her.  It wasn't
that she was jealous that Lauktk and I were close, or had gone on
'adventures' together, it was that we couldn't share them with her. To
not be able to share some of the most formative moments of the
friendship between two people, both of whom she held dear - as a
link-phile, not being able to share like that, it never sat right with
her. She could understand it, rationalize it, accept it in the higher
regions of her mind, but she could never be comfortable with it. She
longed for the day when enough would be declassified that the three of
us could go out to the cabin and just spend a few days reveling in
freedom from the secrets we'd been carrying.  You can't wear a link
and like secrets, it doesn't work that way.  Even carrying the small
ones entrusted to one such as myself, I gained great respect for the
burdens that high command places upon itself. But she never had that
load to bear. She only got to see it second hand, feeling the locks
and filters in our minds that precluded us from divulging, whose
presence trespassed upon her familiarity with each of us.  The cabin,
the three of us - that'll never happen now - and even if the locks all
cleared today, whatever tales I spin can never truly be his story." He
stares directly at the barkeep.  "But as it is, with secrets
unreleased, I am crippled in my ability to even relate the experiences
comprising our time together. At least Mai has her other two
bond-mates to help her through this. I... I have only the ship, as
fresh recovered from her time as an invalid as I mine."

His head lowers and an aura of tired resignation encroaches upon his
face.  "There's a certain chill sometimes, when I can't escape
pondering past, present, and the difference between them.  When my
hands brush across the weld lines that make the paint cringe in the
subtle dismay of a suture, when I see adverts for a new jump drive,
when some phantom process in my brain convinces me I can still smell
the kt'tothan leather that no longer covers the passenger seats, as I
sit here staring at two untouched drinks - I cannot outrun reality,
and I am left with the knowledge that I pilot a dead man's ship.  This
cold clings to one's skin, like some sort of shrink-wrapped leprosy,
and I can't help but wonder if it's going to follow me to every ship
that I'll ever fly." His tone now shifts, with nostalgia being
overshadowed by pained cynicism. "For I can't stop being a pilot - she
beckons, you see; the cold vacuum of space moves her lifeless arm, and
I am compelled to join the legions of ships that, mast-less, sail upon
her.  Days at a time, scant body-lengths away from her condemning kiss
- and there he'll be - a ghost from selves by then long past, waiting
for me to join him.  How many times can I refuse before I am forced to
fold?  How many times can I wake from nightmares to find myself still
sane?" He extends his arm, grasping the water in front of him, and
drains half the vessel before gingerly relinquishing his possession of
the glass, his thirst and anger temporarily quenched, one more
thoroughly than the other. He continues, more calmly than before, "Far
more, I hope, than my melodramatic flares would lead you to
believe. I've no urge to dive into those sunless depths.  Happy or no,
I have it better than many, no worse than most. After all, I was
raised by Klk'k; I've had more than enough time to outgrow pitying
myself merely for being human."

Glimmers of genuine physical exhaustion begin to work their way upon
his face, upon his previously impeccable posture. He leans forward,
right arm upon the bar, supporting his furrowed brow with his right
hand, his splayed fingers covering half his face. "However, even
beyond the shallow realm of self-pity, once you've felt the inexorable
grind of the universe's apathy ... it's hard to see much meaning.  You
can't find a way to convince me that there's any particular reason
that I'm sitting here talking instead of him. Fate has no weavers,
only the mad spider of chance and the ever-spinning spindle of time.
Underneath it all we're not even pawns - pawns can become knights or
queens - we're nameless particles in some sick, twisted Brownian
motion colliding every now and again with each other and changing.  We
dream up gods to play with us, if only so we can pretend to be
pawns. We sup on hubris so that we can aspire to have names.  It is
only a question of which dish we choose to partake of. Do we follow
the Shapers and seek to assault the glass ceiling of perfection
without even the knowledge as to what that would mean?  Do we cloak
ourselves with the counter-empirical idealism of the Andolians,
believing that all problems can be solved, and that our ability to
solve will progress indefinitely?  One could retreat to the scared
futility of the Purist's status quo, or, joining the Unadorned or the
Mechanists, give up the pretense of desiring to be human.  Is there
any solace in the Merchants' proud valuing of wealth or the
High-Born's pride in their idiosyncratic conception of nobility? Hell,
one could give up on humanity and go live with the Shmrn, the Uln, or
the Rlaan-Briin. Despite every oath I've sworn, for all that I do care
about the Protectorate, I've never been able to believe - not the way
he could. Not in a way that mattered.

"When I was a child, the dreams of my parents and their bond-set
sustained me."  He smirks. "Those who didn't know better would think
it bizarre to be sustained by Klk'k dreams, given that they don't have
the whole unconscious labyrinth experience during sleep cycles that
humans do.  There isn't a word for such things in any of the native
languages - no, the Klk'k dream is the daydream, the fantasy, the
meanderings of desires projected onto unfolding futures. Strong dreams
that one can feast on.  As I grew older, the dreams of the societies I
lived in sustained me, and when I met Lauktk, his dream consumed
me. Now I guess I have to make my own... but, as crazy as it sounds, I
feel the need to finish his somehow, and without him... I don't know
what that even means, and even less how I'm going to do it. Indeed, as
clouded as any future seems right now, maybe I shouldn't be surprised
the only dreams I find myself capable of having are nightmares."

He is silent for a few moments, perhaps reliving some fragment of a
nightmare, perhaps preparing his next utterance, perhaps both.  His
face, as still as carved granite and shadowed behind his hand, is
impossible for the barkeep to read. Returning once again to the realm
of motion, he chuckles briefly, in a despondent fashion, and
sighs. For the first time, the motions of his breathing edge into the
realm of mundane perception. He rallies his voice one last time, too
tired for any emotion born of anger to dominate his speech.

"Now, I bet you think this is all just a facade. I bet you think it's
revenge, or anger, or some such that motivates me: that I rage inside
with a desire to kill the Luddites who assaulted us, who drove us
down, who forced me to crash- land, whose actions resulted in
Lauktk's" he falters ever so slightly, "death. I felt that briefly
then, but I feel almost nothing about them now - they aren't important
enough to warrant personal hatred. Pirates, the ISO, the Luddites,
even the Aera - they're all just dancing to the blood rhythms that
cause every cell in their bodies to join in a choral chant of 'Stay
alive! Stay alive!' I can't really blame them for it, even if they'll
probably blame me if I take the lead in the dance and reciprocate
their violence. Pity really. Things would be so much more pleasant if
we could learn to not step on each others' toes, or claws as the case
may be." He turns his gaze up again, letting his fingers fall away
from his eyes, to rest once more on the bar. "Don't think I don't
mourn my friend. It's just that the blackness of an executioner's mask
doesn't make it fit for mourning clothes. If I see the Luddites that
killed him ... I'll probably try to kill them, but for the reason that
they'll be trying to kill me.  Here's my advice, Mr. Robo-barkeep:
don't hold grudges, don't look for comforting answers and don't wait
for magic wands. The first can only hold you back, the second are
never what you want them to be, and the third are always being held by
something that's going to turn you into a toad if you aren't
careful. Feel free to take it with a few grains of salt though. I'm
not anyone qualified to pontificate - me, I'm just someone" he glances
at the still untouched ale "not having a drink at your bar and
... flying a dead man's ship."

[The man finishes his water and leaves, each step clearly a source of
pain from injuries not yet entirely healed.]

\section{Little-Plot segments}
\subsection{The intro plot}

Lauktk's siblings (technically, they aren't all his full brothers and
sisters, as Klk'k families are a bit more complicated than the Western
notion of the nuclear family, but from the Klk'k perspective, they
might as well be) show up in Cephid 17.  Conversing with them about
Lauktk's never to be realized dreams of finding his own personal
truths of freedom in traversing space as his ancestors traversed the
seas of Ktah, they decide the only fit course of action is to have a
prolonged, multi-system funeral commemoration (think "wake") that
takes the remains of Lauktk and his ship to some of the great
landmarks and capitals of friendly space before returning his ashes to
Deucalion's sister and scattering them over the oceans of Ktah.

Implementation issues:

To do this will require the purchasing of a jump drive and some
(perhaps a placeholder for the moment, or even just remove some cargo
space) passenger holding space for the siblings.  The idea will be to
push the player to get enough money to leave the system, and as soon
as they can leave the system, to not just wander aimlessly, but to
force them to see some of the unique places we're going to make. Also,
while on the trip, one should be pushed to take on some sets of fixer
missions that lie on the same path where there are moral/ethical
issues involved as to which of the fixers' missions one takes (thereby
defining somewhat the path the PC will later take, be it more
mercenary, more mercantile, a return to military service, a life of
piracy, or whatever), such that, by the time one returns to Ktah, one
must have enough money that one can buy a new ship (namely, if
possible to script, the original ship should be put aside in deference
to Lauktk), thus completing a rebirth both of the player's psyche,
and, through the change of ship, the rebirth of the player's avatar.

{\bf EVENTS IN ACT 1: Hiroshima (mon ami)}
\begin{itemize}
\item Deucalion engages in side quests for cheap jump drive (optional)
\item Deucalion and Lauktk's bond-siblings take part in wake that canvases friendly space
\item Character development (profession) of Deucalion via encountering certain fixers along the way and interacting with them

 Profession paths:
\begin{itemize}
\item		Officer (Different paths depending on which navy)
\item		Explorer
\item		Pirate
\item		Assassin
\item		Bounty Hunter
\item		Merchant
\item		Tycoon
\item		Fixer (multiple ethical subpaths)
\item		Spook (multiple ethical/political subpaths)
\item		{NEED MORE - ADDME}
\end{itemize}
\item Deucalion meets up with another former service mate, Mirabel, at
the ceremony on Ktah.  Depending upon actions taken during this
chapter, he will either be told that he can't go back to the military,
entreated to come back to the (Andolian Protectorate military, or
questioned as to his interest in coming back.
\item Likewise dependent upon actions, Deucalion may receive a
communique from interested parties of questionable intent/methodology.
\end{itemize}


% LocalWords:  Dramatis Deucalion Klk'k Ktah Kubernan Xenolinguistics APSWAK
% LocalWords:  Andolian Lauktk Deucalion's Purth UniServe Lauktk's Aera Amakakt
% LocalWords:  coreward K'hoeama Lightbearers Oolak'kl Rlaan kt'tothan Shmrn
% LocalWords:  Andolians Mechanists Uln Briin

\section{Klk'k History Books}

\subsection{Excerpts from the Ivan Kltakln Guidebook}

[From “The Klk’k guide to one’s interstellar locale” 12th ed. Ivan Kltakln tr. J. Valthorpe  
catalogued Andolian central distribution, 3263]

Space: noun
1. That which is between objects of interest.
2. A rather large, dark, and primarily empty expanse where, contrary to one’s initial fears, one is exceptionally unlikely to be eaten by a grue.
3. The void in which one must travel if one does not wish to spend one’s existence wandering around the underdeveloped worlds of minor political entities.

Rlaan: noun
1. Humanized representation of the name used by the pair of species from the 4th planet in system SCx9362.
2.	Either of the two species from SCx9362. Rlaan are radialy symmetric beings with a base four split. Their workers stand about one meter high at the prime knee, and are nearly one meter in diameter. Members of their warrior caste tend towards being 50% larger in both dimensions. Rlaan natively breathe a methane-based atmosphere, and must wear special breathing apparatus to negotiate oxygen-nitrogen environments. Their skeletal structure, being an exoskeletal carapace supported internally by millions of reinforcing struts, is best suited to lower gravity worlds, and leads to the use of mechanical assistance on larger or denser rocky bodies. What is reported of Rlaan culture appears to be a rather dry affair, and their music has been compared to the set of frequencies one would expect to register if a Myztherian Octpanther were let loose in a campanile. On a more disturbing note, the Rlaan central archives possess the largest collection of data concerning Jerry Lewis and Yoko Ono outside of Human space.  The Rlaan are, however, regarded by many of the other space faring races as much more intelligent than their culture’s taste in art would suggest.

Aera: noun
1. 	A member of an intelligent centauroid species from some misbegotten hell of a jungle world orbiting SCx62381. The Aera are oxygen-nitrogen breathers, with a strong internal skeleton, smooth, ashen-gray leathery skin, a decided lack of psychiatric assistance for their obviously repressed dissatisfaction with natural ecology, and, at least according to the Cult of the Devourer on Mishtal Seven, a flavor remarkably similar to that of a human with a high protein diet, but only if both have been served with a nice Chianti. 

\subsection{A Brief History in Time and Space}

“In the beginning, all was NULL – or perhaps it was (void*) – we’ll never really know, so we may as well stop worrying about it and get back to obsessing over how we’d rather be fornicating.” [From “The Klk’k guide to one’s interstellar locale” 12th ed. Ivan Kltakln tr. J. Valthorpe catalogued Andolian central distribution, 3263]

1. A REALLY LONG TIME AGO

The beginnings of the universe are especially interesting to physicists, but not to xenopologists, if only because the lack of heavier elements made most forms of life and any recognizable civilization impossible.

2. Some Supernovas and Several Billion Years Later

Somewhere between 12 and 40 million years ago the first interstellar civilization(s?) wandered out from core-ward into this area of the galaxy. At least, this is the current theory spawned from research on the records of the Ancients. No direct data concerning the existence of these beings has yet been found. Based on a passage from what appears to be a historical text found on the Uln homeworld, these predecessors of the Ancients have become known as ‘those who have only names’. This is believed to be a reference to either the homogeneity of their civilization, or to some practice of personification of each of this group’s viewpoints. Detractors to this whole area of research point to this naming scheme as further proof that what was uncovered was not a history text, and instead hold to the view that it was more likely a cheap sci-fi novel.

3. Some Few Million Years Later

While there is much contention about the nature of the predecessors of the Ancients, the Ancients themselves left enough rubble strewn around the galactic arm to convince even a fairly hardened skeptic of their having dwelled in these parts. The Ancients appear to have been made up of at least two major species groups, and interacted with at least three others, albeit it is not known whether these were client species, or contemporaries from another part of the galaxy. Their reign over this region lasted until about 1 to 2 million years ago, whereupon they rapidly ceased to be present. There is a wealth of evidence that severe infighting played some part in the destruction of the Ancients, but, assuming there were victors in such a conflict, little is known of what became of them. The best source of such evidence, however limited, is the Uln homeworld. While they are quite sensitive about the subject, the widely held belief among the major races is that the Uln are the descendants of the Ancient’s equivalents of lab monkeys. The Uln culture sprang up among the remains of a sprawling set of Ancient structures, and advanced in technology faster than their biology or social structures could adapt, leading one noted human researcher to note upon seeing them, ‘It was as if I had suddenly come across a spacecraft piloted by Homo Erectus – if they hadn’t been so ill prepared for the gifts they unintentionally received, they would have conquered the entire arm’. Fortunately for the aspirations of dominance held by other species, the Uln were decidedly unprepared. Indeed, they spent so much time blowing each other up with weapons they didn’t entirely control that it is a wonder that either they or the ruins on their planet still survive.

The ruins, however, did not escape unscathed from the genesis of the Uln culture. While assuredly the largest known source of information on the Ancients, the ruins deliver little coherent information about many key aspects of the Ancients’ existences, largely due to vast portions of many buildings having been turned to dust.

4. A Few Thousand Years Ago

The histories of the Humans, Rlaan, and Aera are best begun in this more recent time period. We shall address them in chronological order:

The Rlaan

The Rlaan would be an interesting study even if they had never achieved space flight. Alone among the known sentient groups, the Rlaan are naturally composed of multiple, in this case, two, species. The speciation of the defender and worker casts appears fairly recent in biological terms, as the two species can, and on not rare occasion do, mate to produce viable, but sterile, offspring. This split between the hunter and gatherer, military and civilian, is a major thread in Rlaan culture, and cannot be overlooked in any attempt to understand them. The Rlaan civilization was the first of the major space-faring races to venture to the stars, and they did so in typically methodical Rlaan fashion, spreading out systematically from their homeworld in sub-light vessels. The period between the beginnings of the Rlaan Diaspora and the Rlaan development of FTL travel was quite long, and it is a testimony to the stability and homogeneity of Rlaan culture that there was precious little cultural drift between the mother and daughter colonies over the centuries before FTL travel, leaving the only noticeable changes between the colonies technological developments and local eccentricities. With the introduction of FTL came the second wave of Rlaan expansion, and the Rlaan’s first encounter with intelligent life, in the form of the primitive Saahasayaay. The Rlaan took the stone-age tribes as their first client race, and managed to control their advancement enough to keep them from obliterating themselves, all without being harsh enough to inspire intense resentment on the part of their clients. The limitations on their inclusion in the power structure of the Rlaan Assembly were deemed insignificant when compared to the gifts of advanced technology that the Rlaan brought to their first clients. The situation with the Rlaan’s other two client species, the Lmpl and Nuhln is even more one of admiration rather than insurrection, as these two species are the result of Rlaan experiments in adapting sub-sentient species to environments that are inhospitable to Rlaan workers, and have been bred for maximum psychological pliability. The Rlaan continued their expansion slowly, but unimpeded, until their first meetings with the Humans.

The Humans

Humanity has, throughout its history, been a balkanized organization. Starting with tribes whose members numbered in the dozens, and moving on to nation-states with millions of adherents, the human homeworld never knew the rule of a single culture. Having moved out of an age of industry and into an age of information and communication, many hoped there would come to be some great alloying of the myriad thoughts of humankind into a coherent culture. While the opening of the world to all sort and manner of information exchange did render the nation-state obsolete, it merely produced a different form of balkanization. Free to contact any other human on their planet, membership in a community came to rely far less on physical location and became more an issue of shared worldview. Unfortunately for the utopians among the humans, these viewpoints were too divergent to alloy. Even as humanity took to the stars, it remained an uneven mix of groups nearly as willing to obliterate each other as to assist any other group. So it was that the first colonies were financed and defined by members of one of several major factions on the homeworld, and their daughter colonies likewise. When strides were made by the Andolians and the Unadorned, whose ability to share data was somewhat hampered by the 40 year delay in data from either side, toward the development of FTL travel, those groups which had already made it to the stars secured their place in power at the expense of those colonists who were still in sub-light transit. Upon their arrivals many decades after their launches, this last unfortunate wave of sub-light travelers found themselves unwanted visitors to their intended homes, hopelessly out of synch with the cultures which had developed, and became known as the Forsaken, banding together on less hospitable worlds to preserve what was left of their identity.

FTL brought humanity into both its first contact with other intelligent forms of life, and into its first fratricidal dispute in interstellar space. The first human group to meet intelligent life was the Unadorned. They encountered a species they named the Mishtali who were enjoying a prolonged and happy bronze age. The Unadorned treated them with a benign neglect, which, given the cultural oddities of both the Unadorned, who come close to religious reverence in their views on computers, and the Mishtali, known for being the source of the ‘cult of the devourer’ wherein the religious rituals are accompanied by the consumption of the remains of alien sentients, is believed by many to have been just as well. Other client races did not fare so well. The pre-sentient Dgn, altered by the Shaper and Lightbearer factions into the modern Dgn and Shmrn, were little better than slaves. While the Shapers made no pretense of treating the Dgn with anything resembling equality, they did so with some measure of respect and without cruelty. The same could not be said for their more extreme brethren the Lightbearers, who, believing themselves to be the forefront of evolution in the entire galaxy, sought to claim their place at the throne of all sentients.

Unfortunately for the Klk’k, it was the Lightbearers who first found them. The Klk’k are unique among human client races in that they had already reached a technological level similar to that of humanity in the mid-twentieth century CE when they were discovered. This level of advancement, however, was of little concern to the Lightbearers, and they set about subjugating the Klk’k to further the glory of humanity. Fortunately for the Klk’k, the Andolians discovered them only some weeks after the first Lightbearer pacification vessel had arrived at the planet. While the treatment of the Dgn had not sat well with the Andolian population, they did not feel it their place to interfere in the experiments of another faction on a species it had created. The Klk’k, however, were existent sentients possessing obvious culture and technology. The Andolians were outraged, and, sending one of the three exploration craft that had entered the system back to Andolian space, proceeded to evacuate to their landing craft and autopilot their vessels into the Lightbearer pacification troop transport and the largest of its escorts. As few spacecraft were military in the modern sense, this was the only way for the Andolians to disrupt the occupation, leaving the Lightbearer troops short on re-supply while the Andolians slowed the pacification troops with guerilla tactics.

Such events, of course, were quickly noticed by both the Andolian and Lightbearer governments, and the first interstellar war in human history began. Although initially fought over control of the Klk’k homeworld, the twin realizations on the part of the Andolians as to the degree to which their industrial sector produced more and better military vessels and as to the degree that the Lightbearers had been willing to destroy the Klk’k caused the Andolians to expand the goals of their military action. The Andolians swarmed through Lightbearer space, destroying nearly all of the Lightbearer’s military craft and unearthing a secret not even known to the Lightbearer’s allies, namely, the existence of the spaceborn, a genetically engineered slave race of humans, designed to live their lives in zero gravity so as to prevent the Lightbearers from having to deal with such menial tasks as laboring in vacuum. It was this revelation that is believed to be most responsible for the lack of action taken by any other faction when the Andolians proceeded to eliminate not only the industrial capacity of the Lightbearers, but also the Lightbearers themselves. Those who did not manage to escape to Shaper or Highborn space, or were not fortunate enough to be killed in the assaults on their worlds, had the dubious honor of being turned over to the Klk’k, the Spaceborn, and the Shmrn. While fewer than expected Lightbearers were killed by those they had abused, the combination of sterilization and incarceration served to eliminate the Lightbearer meme from the realm of dominant thought. These drastic events caused the major factions to invest some effort into the construction of a loose federation of all human colonies, with the major purpose not the advancement of mankind, but a policing against fratricide and a forum for the airing of grievances. Even still, the extreme balkanization of the human factions has rendered the Confederation slow to act and somewhat impotent against all except the most dire of situations. Nonetheless, by the time the Humans and Rlaan met each other, the Confederation had been in existence long enough to act with speaking authority for nearly all of the human colonies, and, despite some initial tensions due to the expansionist natures of both Human and Rlaan, the two entities have enjoyed a calm, if politically charged relationship that has benefited all species involved.

The Aera

Woe betide a lesser race that its birthplace should be so unlike to a cradle. The Aera homeworld was a hellish jungle, and the Aera, like all species it begat, are superb survivors, having been adapted by evolution in a harsh environment. It is the misery of the jungle which has most profoundly shaped Aera culture. From the dawn of their civilization, the Aera have struggled to keep the jungle at bay. They embraced technology in all its forms, from its earliest incarnations as fire and axe, to the laser and crop duster, provided it could beat back whatever jungle that invention’s day held in store. However, the eventual Aera conquest over the jungles of their planet did not remove the mentality that had allowed their civilization to not be swallowed whole over the last few thousand years. The Aera went into space and space became the new jungle – a realm to be watched and controlled, lest from it come erosion of some new sort.

It is their deep comfort level with changing technology combined with their solid organizational patterns and assertive and wary nature which has allowed the Aera to advance to a point arguably beyond that of the Humans or of the Rlaan. Indeed, the Aera were sufficiently precocious as to have developed FTL travel before leaving their home system. However, even these great leaps and advances did not give the Aera what they could never have known they needed – a good position in the local jump topology. When the Aera met the Rlaan and then, soon after, the Humans, they found that these two groups blocked all hope of meaningful expansion core-ward. As diplomatic solutions to the passage of Aera fleets through the core worlds of the Humans and Rlaan were rejected, tensions mounted. Eventually, the Aera made an attempt to bypass Rlaan space with a long range point-jump vessel. However, due to less than forthright information on the extent of their empire, the vessel ended up in the middle of Rlaan space rather than bypassing it, and the effect of having even a lone Aera military vessel deep in Rlaan civilian space was to provoke a military conflict along the Aera/Rlaan border. At nearly the same time, the unfortunate Shmrn, having been granted a colony ship by the Andolians, nearly a century and a half previously, found that they had picked the wrong planet to settle on, as they now found themselves sharing a border with the Aera. This then prompted tensions between the Humans and Aera that, when the Aera realized that there was not going to be a diplomatic solution to their geographical dilemma, also erupted into armed conflict, with the Aera determined to punch through a corridor to core-ward expansion.

5. More or Less Now

The Rlaan and Aera are at a stalemate on the border, with both sides closing towards an unofficial cease-fire. The scaling back of hostilities along said border will no doubt increase the hostilities along the Aera/Human border. The Confed took quite a beating trying to defend its newly founded border worlds, and lost several of them, but has held fairly steady at all of the older systems. An Andolian task force was sent to defend the Shmrn, but the system became surrounded when the forces in the two connecting systems were both overrun. The status of the task force and the system are unknown. Also unknown is how well the Confed will hold up to an Aera assault on a single front. The Rlaan are unlikely to lend any serious assistance, given their need to maintain their own defensive front, and, although they would never acknowledge it, their desire to expand while the Humans are occupied. It is generally believed that the Rlaan both find unsatisfactory the Aera treatment of civilians (likely a result of the Aera having precious few civilians) and fear the Aera’s aggressive tendencies too much to allow them to defeat Humanity, but they appear entirely willing to let them bloody the Confederacy to the point where its economy is entirely sunk into preserving its existence.

